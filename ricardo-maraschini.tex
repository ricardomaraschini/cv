\documentclass{article}
\usepackage{multicol}
\setlength{\columnsep}{1cm}
\usepackage[a4paper, total={7in, 10in}]{geometry}

\begin{document}
\begin{multicols}{2}

\section{Ricardo Maraschini}
ricardo.maraschini@gmail.com\\
Brazilian, 34 years old, Married\\
Currently living in Porto Alegre, Brazil\\
https://github.com/ricardomaraschini

\section{Profile}
I am a seasoned Software Architect with strong skills on Object Oriented
development and Development Methodologies. I've been developing and maintaining
network monitoring solutions on the last 10 years and currently I am
working as Software Architect and Backend Developer for a Startup based on
Amsterdam.

\section{Skills}
\subsection{Programming Languages}
I have expertise in PHP, C, Javascript, Shell Script, Perl and intermediate
skills in Python. Have experience on mobile application development using Apache 
Cordova, PhoneGap, WebSockets, ZMQ, Node.JS, jQuery and jQuery-mobile. Regarding
development methodologies, I've been using SCRUM on last 4 years, and I am an 
enthusiast of GOF Design Patterns and SOAP/Restful APIs.  

\subsection{Debugging tools}
Experience with tools like \textit{GDB} and \textit{strace} for userspace 
debugging and I have already done PHP code profiling with \textit{xdebug}.

\subsection{Databases}
I am experienced on MySQL/MariaDB database design, management and multi-master
environment configuration. I also have done some projects using MongoDB, CouchDB
and SQLite.

\subsection{Version control systems}
Experience with Git, Subversion and CVS. I have also done post commit scripts
implementation to fully integrate Version Control Softwares with Continuous
Building software (\textit{Jenkins}) and ticket management systems 
(\textit{Redmine}). 

\subsection{Operating Systems}
I have strong knowledge on Linux operating system including RHEL, CentOS,
Fedora, Debian and Ubuntu. I'm very curious about the inner workings of 
Operating Systems in general and have also studied how the Linux Kernel
works. I used to play around with OS and you may find some of my C and
Assembly codes on the matter at my GitHub page.

\subsection{Virtualization}
I am experienced on KVM based virtualization, including oVirt
and libvirt/virsh/virt-manager. I also have done integration between oVirt
and Jenkins in order to create and spin up a virtual machine as part of 
the Continuos Build process using libvirt/Python, I have also done OpenStack
AllInOne deployment creating the Development Team's SDN;

\subsection{Storage and Filesystems}
I have experience on setup and manage of filesystems and storage devices
including XFS, NFS and GlusterFS.

\subsection{Network/Protocols}
TCP/IP protocol suite and vast experience on SNMP and related Protocols. Basic 
routing and Firewall configuration(iptables).

\subsection{Containers}
Creation and management of Docker containers in order to keep
homogeneous development environments avoiding further deployment's pains.

\subsection{Configuration Management}
Manage orchestration using Puppet and Ansible is something that
I started to do recently to keep a small farm of building servers
up to date and with equal configurations.

\section{Work experience}
\subsection{Software Architect/Developer}
Watt-Now\textsuperscript{1} Amsterdam, Netherlands, since March 2016. Working coding
and designing the backend of a Cloud Based application using technologies as Docker,
MongoDB, Node.JS and PHP.

\subsection{Software Architect}
OpServices IT\textsuperscript{2} Porto Alegre, Brazil, from Jun 2005 to February 
2016. I have been working as Software Architect of Monitoring products based on Linux
and open source tools. I was also the lead architect and developer working on the 
research and development of monitoring tools using PHP, C, Perl, Python and Shell 
Script, as well the design and maintenance of MySQL databases.

\subsection{System Adminstrator}
TcheTurbo ISP\textsuperscript{3} Frederico Westphalen, Brazil, from Jan 2002 to Jun 2005.
I was responsible for design, deployment and maintenance of firewalls based on
Linux Netfilter/IPTables and servers running MTA (Sendmail and Postfix), Proxy
(Squid) and web servers (Apache) among others.

\section{Education}
\begin{enumerate}
\item Computer Science at URI(graduated in 2005).
\end{enumerate}

\section{Certifications}
\begin{enumerate}
\item ITIL Foundation V3 Certified.
\item LPI Certified
\end{enumerate}

\section{Languages}
\begin{enumerate}
\item English: Working proficiency.
\end{enumerate}

\section{FOSS related activities}

\subsection{Network Cockpit\textsuperscript{4}}
Network Cockpit is a Mobile Application that allow users to see, in real time,
events happening on a Network. It is composed by three separated entities, a
broker, a middleware and the client application and is fully integrated with
monitoring frameworks as Nagios\texttrademark and Naemon. Developed using
Node.JS, C, ZMQ and Apache Cordova.

\subsection{NADA\textsuperscript{5}}
NADA is a project that intents to insert baseline adaptive thresholds to 
Nagios\texttrademark or Naemon monitoring frameworks. Developed using C,
MySQL/MariaDB and SQLite.

\subsection{Contributions}
\begin{itemize}
	\item Nagios
		\begin{itemize}
			\item http://www.nagios.org
		\end{itemize}
	\item Nagios Exchange
		\begin{itemize}
			\item http://exchange.nagios.org
		\end{itemize}
	\item Naemon
		\begin{itemize}
			\item http://www.naemon.org
		\end{itemize}
	\item modgearman
		\begin{itemize}
			\item https://labs.consol.de/nagios/mod-gearman
		\end{itemize}
\end{itemize}

\section{References}

\tiny \textsuperscript{1} http://www.watt-now.nl\\
\tiny \textsuperscript{2} http://www.opservices.com.br\\
\tiny \textsuperscript{3} http://www.tcheturbo.com.br\\
\tiny \textsuperscript{4} https://github.com/ricardomaraschini/network-cockpit/wiki\\
\tiny \textsuperscript{5} https://github.com/ricardomaraschini/nada

\end{multicols}
\end{document}
